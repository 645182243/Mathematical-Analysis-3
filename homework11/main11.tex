

\subsection{P26}
\begin{enumerate}[1]
\item[5]
    \begin{enumerate}[(i)]
    \item
    $|f(\bm{x})-f(\bm{y})|=0 \Rightarrow |\bm{x}-\bm{y}|=0 $\qed

    \item
    $|Df \cdot \dif \bm{x}|=|\dif \bm{y}| \geqslant c|\dif\bm{x}|$
    \par $\therefore$ the absolute values of $Df$'s eigenvalues are not smaller than c \qed

    \item
    Use reduction to absurdity. Let $f(\mathbb{R}^n)\neq \mathbb{R}^n$, it is evident that $f(\mathbb{R}^n)$ has a boundary point which is not an isolated point.
    \par Denote $\bm{y}_b$ as a boundary point of $f(\mathbb{R}^n)$ with a neighboring point $\bm{y}_0\in f(\mathbb{R}^n)$ and $|\bm{y}_b - \bm{y}_0|=\varepsilon$
    \par $\therefore$ $\exists \bm{x}_0, f(\bm{x}_0)=\bm{y}_0$
    \par $\because$ $\exists U\in \mathbb{R}^n, |\partial U-\bm{x}_0|>\frac{2\varepsilon}{c}$
    \par $\therefore$ $|\partial f(U)-\bm{y}_0|>2\varepsilon$
    \par $\therefore$ $\bm{y}_b$ is the interior point of $f(U)$
    \par $\therefore$ contradicts to the fact that $\bm{y}_b$ is a boundary point \qed
    \end{enumerate}
\end{enumerate}

\subsection{p79}
\begin{enumerate}[3]
    \item
    For $\forall \varepsilon >0 $, the accumulation points can be covered by a finite set of closed rectangles with total volume $V_0 <\frac{\varepsilon}{2}$
    \par $\because$ If there are infinite points left in $S$, there will be an arbitrarily small region containing infinite points in $S$ outside previous rectangles, which contradicts to the fact that all accumulation points are already covered.
    \par $\therefore$ The left points could be covered by another finite set of closed rectangles with total volume $V_1 <\frac{\varepsilon}{2}$ \qed
\end{enumerate}

\subsection{p88}
\begin{enumerate}[4]
    \item
    Negative.
    \par Let $f_n(x)=[R(x)]^{\frac{1}{n}}\rightarrow D(x)$, $R(x)$ is Riemann function and $D(x)$ is Dirichlet function
    \par Evidently $f_n(x)$ is integrable while $D(x)$ is not \qed
\end{enumerate}