\usepackage{natbib}
\usepackage{url}
\usepackage{amsmath}
\usepackage{graphicx}
\graphicspath{{images/}}
\usepackage{parskip}
\usepackage{fancyhdr}
\usepackage{commath}%定义d
%\usepackage[UTF8]{ctex}
\usepackage{geometry}   
\usepackage{titlesec}
\usepackage{caption}
\usepackage{paralist}
\usepackage{multirow}
\usepackage{booktabs} % To thicken table lines
\usepackage{diagbox}
\usepackage{authblk}
\usepackage{indentfirst}
\usepackage{float}
\usepackage{amsthm}
\usepackage{fontspec}
\usepackage{color}
\usepackage[perpage]{footmisc}%脚注每页清零
%\usepackage{txfonts} %设置字体为times new roman
\usepackage{lettrine}
\usepackage{nameref}
%\usepackage[nottoc]{tocbibind}
\usepackage{amssymb}%font
\usepackage{lipsum}%make test words
\usepackage{picinpar}%words around the picture
\usepackage[all]{xy}%draw arrow
\usepackage{asymptote}%draw picture
\usepackage{fontawesome}
\usepackage{titletoc}
\usepackage{fourier-orns}
\usepackage{nameref}
\geometry{bottom=4cm}

\newcommand{\sgn}{\mathop{\mathrm{sgn}}}


\pagestyle{fancy}
\fancyhf{}
\cfoot{\thepage}
%\lhead{\bfseries\rightmark}


%\setmainfont{Times New Roman}

%\ctexset{today=small}%日期类型设置

% ======================================
% = Color de la Universidad de Sevilla =
% ======================================
\usepackage{tikz}
\definecolor{PKUred}{RGB}{126,24,28}

%超链接设置
\usepackage[breaklinks,colorlinks,linkcolor=black,citecolor=PKUred,urlcolor=black,pagebackref,bookmarksnumbered]{hyperref}


\newcommand{\hsp}{\hspace{20pt}}

\titleformat{\section}{\LARGE\sffamily\bfseries}{\hspace{-15pt}\textcolor{PKUred}{\vrule width 4pt height 20pt}\hsp\arabic{section}}{10em}{}

\titleformat{\subsection}{\normalfont\large\bfseries}{}{-5pt}{}

\titlecontents{section}[0pt]{\addvspace{1pc}%
\sffamily}%
{\contentsmargin{0pt}%
\bfseries\makebox[0pt][r]{\Large\thecontentslabel\enspace}%
}{}
{} [\addvspace{1pt}]


\titlecontents{subsection}
[0.2em] % ie, 1.5em (chapter) + 2.3em
{}{} {} {\titlerule*[1pc]{.}\contentspage}



\iffalse 
\renewcommand*\footnoterule{%
    \vspace*{-3pt}%
    {\color{PKUred}\hrule width 2in height 0.4pt}%
    \vspace*{2.6pt}%
}
\fi
\renewcommand*\headrule{%
    {\color{PKUred}\hrule width \textwidth height 0pt}%
    \vspace*{2.6pt}%
}


%% Color the bullets of the itemize environment and make the symbol of the third
%% level a diamond instead of an asterisk.
%h\renewcommand*\textbullet{\dag}
\renewcommand*\labelitemi{\color{PKUred}\textbullet}
\renewcommand*\labelitemii{\color{PKUred}--}
\renewcommand*\labelitemiii{\color{PKUred}$\diamond$}
\renewcommand*\labelitemiv{\color{PKUred}\textperiodcentered}



%%% Equation and float numbering
\numberwithin{equation}{section}		% Equationnumbering: section.eq#
\numberwithin{figure}{section}			% Figurenumbering: section.fig#
\numberwithin{table}{section}				% Tablenumbering: section.tab#


%代码设置
\usepackage{listings}
\usepackage{fontspec} % 定制字体
\newfontfamily\menlo{Menlo}
\usepackage{xcolor} % 定制颜色
\definecolor{mygreen}{rgb}{0,0.6,0}
\definecolor{mygray}{rgb}{0.5,0.5,0.5}
\definecolor{mymauve}{rgb}{0.58,0,0.82}
\lstset{ %
backgroundcolor=\color{white},      % choose the background color
basicstyle=\footnotesize\ttfamily,  % size of fonts used for the code
columns=fullflexible,
tabsize=4,
breaklines=true,               % automatic line breaking only at whitespace
captionpos=b,                  % sets the caption-position to bottom
commentstyle=\color{mygreen},  % comment style
escapeinside={\%*}{*)},        % if you want to add LaTeX within your code
keywordstyle=\color{blue},     % keyword style
stringstyle=\color{mymauve}\ttfamily,  % string literal style
frame=single,
rulesepcolor=\color{red!20!green!20!blue!20},
% identifierstyle=\color{red},
language=c++,
xleftmargin=4em,xrightmargin=2em, aboveskip=1em,
framexleftmargin=2em,
numbers=left
}

%脚注
\renewcommand\thefootnote{\fnsymbol{footnote}}

%定义常数i、e、积分符号d
\newcommand\mi{\mathrm{i}}
\newcommand\me{\mathrm{e}}
