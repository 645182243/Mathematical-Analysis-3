\subsection{P117}
\begin{enumerate}[1]
    \item[2] 
    \begin{enumerate}[(1)]
        \item 
        $\{U_{\alpha}\}$ is a open cover of $M,\ \forall U_{\alpha},\ \exists f:U_{\alpha} \mapsto V_{\alpha},\ (x,y,\sqrt{x^2+y^2}) \mapsto (x,y) $
        \par $\because f$ is a bijection, $f$ is continuous, the inverse function $f^{-1}$ is continuous
        \par $ \therefore f $ is a homeomorphism
        \par $\therefore $ M is a  two-dimensional manifold

        \item 
        Assume that the entire cone is a two-dimensional manifold
        \par for open cover $U_{\alpha}\ni (0,0,0),\ \exists $ homeomorphism $ \varphi  : U_{\alpha} \mapsto V_{\alpha} $
        \par $ \because V_{\alpha} \setminus \{\varphi(0,0,0)\} $ is connected, but $U_{\alpha} \setminus \{(0,0,0)\}$ is not 
        \par $\therefore  $ contradict with two homeomorphic spaces share the same topological properties \qed
    \end{enumerate}

    \item[6]
    Assume that $\{U_{\alpha}\}$ is the open cover of $M$, for $\forall U_{\alpha},\ \exists \varphi_{\alpha}: U_{\alpha}\mapsto V_{\alpha}$
    \par $\varphi_{\alpha}:(x_1,\cdots,x_n) \mapsto (x_1,\cdots,x_{n-k},f(x_1,\cdots,x_n)) $
    \par $ \Dif \varphi_{\alpha} = \begin{bmatrix} E_{(n-k)\times (n-k)} & 0_{(n-k)\times k} \\\Dif f_{k\times (1\sim (n-k))} &\Dif f_{k\times ((n-k+1)\sim n )} \end{bmatrix}$
    \par $\because \forall x_0 \in U_{\alpha} $ rank$(\Dif f(x_0) )= k $, assume that rank$(\Dif f(x_0)_{k\times ((n-k+1)\sim n )} )= k $
    \par $\therefore \det \Dif \varphi_{\alpha} \not= 0 $
    \par according to implicit function theorem, $ \exists $ open set $ U \ni x_0 $, open set $ V \ni \varphi_{\alpha}(x_0)$ s.t. $ \varphi_{\alpha}: U \mapsto V$ is a diffeomorphism
    \par $ \varphi_{\alpha}: U\cap M \mapsto V \cap \{(y_1,\cdots,y_n)\in \mathbb{R}^n | y_{n-k+1} = \cdots = y_n=0 \}$
    \par $ \therefore M$ is a (n-k)-dimensional differential manifold\qed
\end{enumerate}

\subsection{P128}

\begin{enumerate}[4]
    \item
    $\omega \wedge \cdot$ is a linear operation which maps $0$ to $0$
    \par $\therefore$ Apparently $M_{\omega}$ is a linear subspace of $V$\qed
    \par  let the first $q$ vectors of $\{e_1 \cdots e_n\}$ be the complete orthonormal basis of $M_{\omega}$
    \par $\because$ for $\omega=\sum\limits_{1\leqslant i_1 < \cdots < i_p \leqslant n} \omega_{i_1 \cdots i_p} e_{i_1}\wedge \cdots \wedge e_{i_p}$ and for $e_k\in \{e_1,\cdots,e_q\},\ \omega \wedge e_k = 0$
    \par $\therefore \delta_{i_1 \cdots i_p k}=0$ for $\forall i_1 \cdots i_p \in  \{i_1, \cdots ,i_p|\omega_{i_1 \cdots i_p}\neq 0\}$
    \par $\therefore$ $k$ must be a common index of all non-zero $\omega_{i_1 \cdots i_p}$
    \par $\therefore$ $q\leqslant p$\qed
    \par $(\Rightarrow):$ under the previous setting, let $q=p$
    \par $\therefore$ $\{1,2,\cdots,p\}$ are all common indices of all non-zero $\omega_{i_1 \cdots i_p}$
    \par $\therefore$ only $\omega_{1\cdots p}\neq 0 \Rightarrow \omega =\omega_{1\cdots p} e_1 \wedge \cdots \wedge e_p$
    \par $(\Leftarrow):$ let $\omega =v_1 \wedge \cdots \wedge v_q$
    \par apparently $\{v_1 \cdots v_p\}$ are linearly independent
    \par $\therefore$ $\{v_1 \cdots v_p\}$ can be the basis of $M_{\omega} $\qed
\end{enumerate}